% Metódy inžinierskej práce

\documentclass[10pt,twoside,slovak,a4paper]{article}

\usepackage[slovak]{babel}
%\usepackage[T1]{fontenc}
\usepackage[IL2]{fontenc} % lepšia sadzba písmena Ľ než v T1
\usepackage[utf8]{inputenc}
\usepackage{graphicx}
\usepackage{url} % príkaz \url na formátovanie URL
\usepackage{hyperref} % odkazy v texte budú aktívne (pri niektorých triedach dokumentov spôsobuje posun textu)

%\usepackage{cite}
%\usepackage{times}

\pagestyle{headings}

\title{Lepší a zdravší životný štýl vďaka gamifikácií\thanks{Semestrálny projekt v predmete Metódy inžinierskej práce, ak. rok 2015/16, vedenie: Ing. Zuzana Špitálová}} % meno a priezvisko vyučujúceho na cvičeniach

\author{Matej Ľoch\\[2pt]
	{\small Slovenská technická univerzita v Bratislave}\\
	{\small Fakulta informatiky a informačných technológií}\\
	{\small \texttt{xloch@stuba.sk}}
	}

\date{\small 6. novembra 2022} % upravte



\begin{document}

\maketitle

\begin{abstract}
\ldots
\end{abstract}



\section{Úvod}





\section{Net Generácia a jej životný štýl} \label{net_generacia_a_jej_zivotny_styl}


Choroby a negatívne zdravotné návyky sú v súčasnosti hlavnými problémami mnohých ľudí na svete, najmä mladých dospelých a detí, ktoré sú súčasťou net generácie. Spôsob, ako môžu títo mladí pacienti zlepšiť svoje zdravie, je zmeniť svoje zdravotné správanie a vykonať dlhodobé zmeny vo svojom životnom štýle. Podľa vedcov však čistá generácia vyvíja kratšie rozpätia pozornosti ako predchádzajúce generácie, čo spôsobuje, že si vytvárajú negatívne zdravotné návyky a sťažuje im zmenu ich životného štýlu.

Ďalšou prekážkou, s ktorou sa títo pacienti budú potýkať, je zhoršený prístup a cenová dostupnosť k zdravotnej starostlivosti. Avšak nové možnosti, ktoré prichádzajú s technologickým vývojom by potenciálne mohli búrať prekážky prístupu k zdravotnej starostlivosti.


\subsection{Net Generácia} \label{net_generacia_a_jej_zivotny_styl:net_generacia}
Je to označenie generácie ľudí, ktorí vyrastali obklopení technológiami, ako sú počítače, televízia, smartfóny, videohry a internet. Existujú určité nezhody, pokiaľ ide o to, koho presne zahŕňa. Podľa atribútov z roku 2004 zahŕňa ľudí narodených v rokoch 1977 až 1997. Avšak neplatí to v globále, keďže nie vo všetkých krajinách bola dostupnosť moderných technológií.

\section{M-Health aplikácie} \label{m_health_aplikacie}
Čoraz častejšie sa stáva gamifikácia populárnejšou témou a vďaka empirickým štúdiám je dokázané, že prináša výsledky. Mnoho doterajších štúdií a viaceré testy preukázali veľký úspech gamifikácie v rôznych sektoroch, a to najmä v oblasti vzdelávania. Avšak v oblasti zdravotníctva chýba výskum gamifikácie. Všetká nádej je však vkladaná do aplikácií, ktoré sú spojené s M-Helath. 

\section{Aplikácie M-Health pre Net Generácia} \label{aplikacie_m_health_pre_net_generaciu}
Vďaka výskumu z roku 2009 vieme niekoľko zistení o funkčnosti riešenia problému pomocou M-Healt aplikácií. Pri otázke „aké konkrétne typy aplikácií m-health používajú študenti a aké typy by nepoužívatelia chceli používať?“ (nepoužívatelia, sú ľudia ktorý nevyužívaju žiadnu M-Health aplikáciu) vyplýva, že fitness aplikácie sa používajú najviac a používatelia, ktorí nie sú používateľmi, by boli najviac otvorení používaniu aplikácií spánku. Fitness aplikácie obsahujú 38\texttt{\%}, zatiaľ čo aplikácie spánku tvoria menej ako 1\texttt{\%}. Preto je pravdepodobné, že existuje nadmerná ponuka fitness aplikácií a nedostatočná ponuka aplikácií spánku. Dôsledkom tohto zistenia je, že je potrebné vyvinúť viac aplikácií m-health súvisiacich so spánkom, aby sa povzbudil väčší počet študentov, aby začali používať aplikácie m-health. Podľa ďalšej otázky, kde sa výskumníci pýtali „ako ovplyvňuje zahrnutie prvkov gamifikácie do aplikácií m-health ich prijatie študentmi?“ sa mnoho mnoho študentov vyjadrilo, že s väčšou pravdepodobnosťou prijmú aplikáciu m-health, ak obsahuje prvky gamifikácie. 

Tento výskum skúmal gamifikáciu v aplikáciách m-Health a jej účinky na zmeny správania študentov v oblasti zdravia. Uskutočnilo sa to prieskumom vzorky študentov z Univerzity v Kapskom Meste a skúmaním vzťahu medzi ich používaním gamifikácie v aplikáciách m-health a tým, ako vnímali, že sa v dôsledku toho zmenilo ich zdravotné správanie. Treba však podotknúť, že tieto závery sú založené na štatisticky významných údajoch, a preto je štatisticky reprezentatívny pre celú populáciu.

\section{Gamifikácia M-Health pre dospelých 50+} \label{pre_dospelych_50}
Svetová zdravotnícka organizácia(WHO) vo svojej správe z roku 2020 uviedla, že v nadchádzajúcich desaťročiach sa svetová populácia nad 60 rokov zvýši zo súčasných 841 miliónov na 2 miliardy do roku 2050, čím sa chronické choroby a blahobyt tretieho veku stanú novými globálnymi výzvami v oblasti verejného zdravia. Podľa WHO je nárast dlhovekosti spôsobený najmä v krajinách s vysokými príjmami, najmä poklesom úmrtí na kardiovaskulárne choroby. Organizácia však upozorňuje, že ľudia nie sú nevyhnutne zdravší, pretože stále predstavujú populáciu, ktorá je náchylnejšia na vážne riziká týkajúce sa chorôb a epidémií. Upozorňuje tiež, že je potrebné zlepšiť prevenciu a riadenie chronických stavov. 

\section{M-Health aplikácie} \label{m_health_aplikacie}
Čoraz častejšie sa stáva gamifikácia populárnejšou témou a vďaka empirickým štúdiám je dokázané, že prináša výsledky. Mnoho doterajších štúdií a viaceré testy preukázali veľký úspech gamifikácie v rôznych sektoroch, a to najmä v oblasti vzdelávania. Avšak v oblasti zdravotníctva chýba výskum gamifikácie. Všetká nádej je však vkladaná do aplikácií, ktoré sú spojené s M-Helath. 

V posledných rokoch sa na trhu objavuje niekoľko gamifikovaných mobilných zdravotníckych aplikácií (m-health) ako prísľub zlepšenia kvality života používateľov ako sľubnej stratégie pre digitálnu zdravotnú starostlivosť. Na podporu lepšieho zapojenia prináša súčasné m-health väčšinou herné prvky používané v tradičných hrách, ktoré pomáhajú používateľom dosiahnuť ich ciele, aby sa motivovali k akýmkoľvek problémom, ktoré majú. V oblasti zdravia, medzi rôznymi prístupmi, či už ako gamifikácia alebo seriózne hry, sú aplikácie zamerané na pohodu pacientov a na lekárske monitorovanie s cieľom pomôcť pri liečbe chorôb, ako je cukrovka, monitorovanie hypertenzie a iné postihnutia.

Avšak väčšina m-health aplikácii na trhu je nie špeciálne navrhnutá pre dospelých vo veku 50+. Tieto aplikácie nezohľadňujú vekovo špecifické zmeny v motivácii hrať alebo vnímania herných prvkov.

\section{Výskum v domove dôchodcov} \label{vyskum_v_domove_dochodcov}
Cieľom výskumu bolo pochopiť úlohu prvkov odmien v platforme m-health gamification pre dospelých 50+ \cite{GamificationElements:Adults}. Aby sa tak stalo, uskutočnili sme workshop s dospelými 50+, aby sme zachytili ich motiváciu a názory na to, či by chceli byť odmenení po ukončení svojich bežných aktivít a ako by sa mohli zamestnať vo svojej rutine.

Z výskumu vyplynulo, že Aj pri dnešnom vysokom percente majiteľov smartfónov ho 3 účastníci stále nemajú. Na otázku o dôvode boli odpovede: "Bojím sa použiť", "Mám hrôzu zo smartfónu" a "Nikdy som ho netestoval".  Spomedzi ostatných účastníkov 70 \texttt{\%} používa aplikácie sociálnych sietí na komunikáciu a sledovanie noviniek.

\section{Výskum Alzheimrovej choroby spojený s hraním hier} \label{vyskum_v_domove_dochodcov}
Výskumníci spolu s hernou spoločnosťou Glitchers  vyvinuli v roku 2017 hru s názvom Sea Hero Quest\cite{Spiers2021ExplainingWV:SeaHeroQuest}. Neskôr vyšla verziu tejto hry aj pre virtuálnu realitu. Princíp spočíva v tom, že sa hráči zmocnia role moreplavca a počas hrania musia naplno využívať svoju orientáciu, postreh a priestorové vnímanie \cite{Virualizing:Conference}. Tieto schopnosti ovplyvňuje Alzheimrova choroba ako prvé. Pohyby a reakcie hráča hra celú dobu zaznamenáva a dokáže poznať aj milisekundové odchýlky. Vedcom tieto informácie dokážu poskytnúť detailný obraz toho, ako vyzerajú reakcie, postreh a priestorové vnímanie ľudí v každom veku podľa pohlavia aj podľa miesta kde žijú. Vďaka týmto informáciám lepšie vedia, kedy je zhoršenie týchto schopností prirodzeným vekom a kedy spôsobené samotnou chorobou. Od spustenia tejto hry si ju už zahralo 3,5 milióna ľudí v 193 krajinách a tým pomohli získať dáta, ktoré sú ekvivalentom 15 000 rokov laboratórneho výskumu. Preto túto hru už v roku 2020 stiahli z obehu, keďže vedci už viac týchto dát nepotrebujú.


\section{VR hry zlepšujú pamäť starším ľuďom} \label{vr_hry_zlepsuju_pamat}
Podľa výskumu Kalifornskej univerzity dokáže hranie VR hier seniorom zlepšiť pamäť a pomôcť tak pri boji s neurodegeneratívnymi chorobami. V tomto výskume hralo 48 seniorov v priemernom veku 68 rokov hru Labyrint VR po dobu 4 týždňov. Účastníci absolvovali testy založené na zapamätanie si objektov a tí, ktorí hru hrali získali významne vyššie skóre, než placebo skupina, ktorá hru nehrala. Bežne sa v populácii nachádza zhruba 5\texttt{\%} seniorov, ktorý majú pamäť porovnateľnú s mladými ľuďmi vo veku vysokoškolákov. Po hraní hry Labyrint VR sa podľa autorov výskumu dostalo 75\texttt{\%} seniorov na túto rovnakú úroveň. Preto výskumníci odporúčajú starším ľudom, aby hrali tieto hry.













\section{Záver} \label{zaver} % prípadne iný variant názvu



%\acknowledgement{Ak niekomu chcete poďakovať\ldots}


% týmto sa generuje zoznam literatúry z obsahu súboru literatura.bib podľa toho, na čo sa v článku odkazujete
\bibliography{my_literatura}
\bibliographystyle{plain} % prípadne alpha, abbrv alebo hociktorý iný
\end{document}
